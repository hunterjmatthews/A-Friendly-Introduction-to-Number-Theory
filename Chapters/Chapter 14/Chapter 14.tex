\documentclass{report}

\input{preamble}
\input{macros}
\input{letterfonts}

\title{\Huge{A Friendly Introduction to Number Theory}\\Chapter 14 Solutions}
\author{\huge{Hunter Matthews}}
\date{}

\begin{document}

\maketitle
\newpage
% -- Begin Chapter 14 -- 
% -- Problem 14.1 -- 
\qs{}{If $a^{n}+1$ is prime for some numbers $a\geq 2$ and $n\geq 1$, show that $n$ must be a power of 2.}
\pf{Proof}{Notice the fact that if $n$ is odd, then $a^{n}+1\mid a+1$. For example, take $2^{3}+1\mid 2+1 = 3$ and $3^{3}+1\mid 3+1 = 7$. Furthermore, let us make the assumption that $n$ is not a power of 2. If this is the case, we can factor it as $n = 2^{k}m$ such that $m\geq 3$ and $m$ is odd. Therefore, $$a^{n}+1\to(a^{2^{k}})^{m}+1\mid a^{2^{k}}+1$$ Hence, $a^{n}+1$ cannot be prime and $n$ must be a power of 2.}
\newpage
% -- Problem 14.2 -- 
\qs{}{ Let $F_{k}=2^{2^{k}}+1$. For example, $F_{1}=5, F_{2}=17, F_{3}=257$, and $F_{4}=65537$. Fermat thought that all the $F_{k}$'s might be prime, but Euler showed in 1732 that $F_{5}$ factors as $641\cdot 6700417$, and in 1880 Landry showed that $F_{6}$ is composite. Primes of the form $F_{k}$ are called \emph{Fermat primes}. Show that if $k\neq m$, then the numbers $F_{k}$ and $F_{m}$ have no common factors; that is, show that gcd($F_{k},F_{m}) = 1$.}
\pf{Proof}{Suppose that $0\leq m < k$ with $F_{m}$ and $F_{k}$ having a common factor $a>1$. Then we can say that $a$ divides both $F_{0}\cdot\cdot\cdot F_{k-1}$ and $F_{k}$. Hence, $a$ divides the difference and $a$ is forced to become $2$ posing a contradiction, because each Fermat number is clearly odd. Therefore, $a$ must be 1 which proves that gcd($F_{k},F_{m}) = 1$}
\newpage
% -- Problem 14.3 -- 
\qs{}{The numbers $3^{n}-1$ are never prime $if n\geq 2$, since they are always even. However, it sometimes happens that $(3^{n}-1)/2$ is prime. For example, $3^{3}-1)/2 = 13$ is prime.\\
\\\textbf{(a)} Find another prime of the form $(3^{n}-1)/2$.\\
\\\textbf{(b)} If $n$ is even, show that $(3^{n}-1)/2$ is always divisible by 4, so it can never be prime.\\
\\\textbf{(c)} Use a similar argument to show that if $n$ is a multiple of 5 then $(3^{n}-1)/2$ is never a prime.\\
\\\textbf{(d)} Do you think that there are infinitely many primes of the form $(3^{n}-1)/2$?
}
\pf{Answer}{\textbf{(a)} Another prime of the form $(3^{n}-1)/2$ would be 13. We can show this by the following computation. $$(3^{3}-1)/2 = 13$$ which is prime.\\
\\\textbf{(b)} Let us assume that $n$ is even, so $n = 2k$. Then we have $$\dfrac{(3^{2k}-1)}{2}= \dfrac{(9^{k}-1)}{2}$$ However, $9k\equiv 1\pmod{8}$, so $(9^{k}-1)/2$ is divisible by 4.\\
\\\textbf{(c)} Let us assume that $n$ is a multiple of 5, so $n = 5k$. Then we have $$\dfrac{(3^{5k}-1)}{2}= \dfrac{(243^{k}-1)}{2}$$ However, $243-1$ is $242 = 2\cdot 11^{2}$, so $$243^{k}=(2\cdot 11^{2}+1)^{k}\equiv 1\pmod{11^{2}}$$ so $(243^{k}-1)/2$ is divisible by $11^{2}$ and thus will never be a prime.\\
\\\textbf{(d)} Although it has yet to be proven, I believe that there are infinitely many primes of the form $(3^{n}-1)/2$.
}
% -- End Chapter 14 -- 
\end{document}
