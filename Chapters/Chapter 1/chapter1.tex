\documentclass{article}
\usepackage[utf8]{inputenc}
\usepackage{amsmath}
\usepackage{listings}
\usepackage{amssymb}

\title{%
  A Friendly Introduction to Number Theory \\
  \large Chapter 1: What is Number Theory? \\
    Solutions}
\author{Hunter Matthews}
\date{09/01/2022}

\begin{document}

\maketitle
\newpage
\tableofcontents
\newpage
\section{Chapter 1: What Is Number Theory?}
\subsection{Exercise 1}
\textbf{Question:} The first two numbers that are both squares and triangles are 1 and 36. Find the next one
and, if possible, the one after that. Can you figure out an efficient way to find triangular square numbers? Do you think that there are infinitely many?\\
\\\textbf{Solution:} You could write a simple program that prints out the first such numbers $n$ such that $n\leq 100000$. You will get 1, 36, 1225, and 41616. While this method will work, it is not the most efficient way to generate triangular-square numbers. An efficient method would be to construct a sequence $\{a_n:\forall n\in\mathbb{N}\}$ which follows a recurrent condition
\begin{equation*}
\begin{split}
a_1 = 1
\\a_2 = 6
\\a_{n+2}=6\cdot a_{n+1}-a_n
\end{split}
\end{equation*}
The square of numbers $a_n$ would be triangular-square numbers. Hence, there are infinitely many such numbers.
\newpage
\subsection{Exercise 2}
\textbf{Question:} Try adding up the first few odd numbers and see if the numbers you get satisfy some sort of pattern. Once you find the pattern, express it as a formula. Give a geometric verification that your formula is correct.\\
\\\textbf{Solution:} For the first few odd n umbers, we have
\begin{equation*}
\begin{split}
1 = 1 = 1^2
\\1 + 3 = 4 = 2^2
\\1 + 3 + 5 = 8 = 3^2
\\\sum_{i = 1}^n (2\cdot i + 1) = n^2
\end{split}
\end{equation*}
To get from the square of $n^2$ to $(n+1)^2$, we need to add two sides of length $n$ and a diagonal element. Therefore, $(n+1)^2-n^2=2\cdot n+1$. If $n^2$ was the sum of the first odd number $n$ then $(n+1)^2$ is the sum of the first odd numbers $n+1$.
\newpage
\subsection{Exercise 3}
\textbf{Question:} The consecutive odd numbers 3, 5, and 7 are all primes. Are there infinitely many such “prime triplets”? That is, are there infinitely many prime numbers $p$ such that $p+2$ and $p+4$ are also primes?\\
\\\textbf{Solution:} If we assume $p$ is not 3, then the remainder of $p$ when divided by 3 is either 1 or 2. Hence, either $p+2$ or $p+4$ is divisible by 3 meaning that there is only one such triple. It has been conjectured that there are infinitely many primes $p$ such that $p+2$ and $p+6$ are prime, but this has yet to be proven. Similarity, it has been conjectured that there are infinitely many primes $p$ such that $p+4$ and $p+6$ are prime, but this has yet to be proven.
\newpage
\subsection{Exercise 4}
\textbf{Question:} It is generally believed that infinitely many primes have the form $N^2+1$, although no one knows for sure.\\
\textbf{(a)} Do you think that there are infinitely many primes of the form $N^2-1$?\\
\textbf{(b)} Do you think that there are infinitely many primes of the form $N^2-2$?\\
\textbf{(c)} How about of the form $N^2-3$? How about $N^2-4$?\\
\textbf{(d)} Which values of a do you think give infinitely many primes of the form $N2-a$\\
\\\textbf{Solution:}\\
\textbf{(a)} Since $N^2-1 = (N-1)(N+1)$, $N^2-1$ will never be prime if $N\geq 2$.\\
\textbf{(b)} Yes, it is very likely that there are infinitely many primes of this form.\\
\textbf{(c)} Yes, it is very likely that there are infinitely many primes of the form $N^2-3$. However, for $N^2-4 = (N-2)\cdot(N+2)$ and is only prime for $N = 3$.\\
\textbf{(d)} If $a$ is a perfect square, say $a = b^2$, then there will not be infinitely many primes of the form $N^2-a$ since we have $N^2-a = N^2-b^2 = (N-b)(N+b)$. Hence, as long as $a$ is not a perfect square, then it is conjectured that there are infinitely many primes.
\newpage
\subsection{Exercise 5}
\textbf{Question:} The following two lines indicate another way to derive the formula for the sum of the first $n$ integers by rearranging the terms in the sum. Fill in the details.
\begin{equation*}
\begin{split}
1+2+3+\cdot\cdot\cdot+n = (1+n)+(2+(n-1))+(3+(n-2))+\cdot\cdot\cdot
\\ = (1+n)+(1+n)+(1+n)+\cdot\cdot\cdot
\end{split}
\end{equation*}
How many copies of $n+1$ are in there in the second line? You may need to consider the cases of odd $n$ and even $n$ separately. If that's not clear, first try writing it out explicitly for $n = 6$ and $n = 7$.\\
\\\textbf{Solution:} Suppose that $n$ is even. Then we get $\dfrac{n}{2}$ copies of $1+n$, so the total is
\begin{equation*}
\dfrac{n}{2}(1+n)=\dfrac{n^2+n}{2}
\end{equation*}
Next, suppose that $n$ is odd. Then we get $\dfrac{n-1}{2}$ copies of $1+n$ and also the middle term $\dfrac{n+1}{2}$ which hasn't yet been counted. To illustrate, let $n = 9$. Then we have
\begin{equation*}
1+2+\cdot\cdot\cdot+9 = (1+9)+(2+8)+(3+7)+(4+6)+5
\end{equation*}
As you can see, we get 4 copies of 10, and 1 copy of 5. Hence, we have 45. For a general $n$, we have
\begin{equation*}
\dfrac{n-1}{2}(1+n)+\dfrac{n+1}{2}=\dfrac{n^{2}-1}{2}+\dfrac{n+1}{2}=\dfrac{n^{2}+n}{2}
\end{equation*}
\newpage
\subsection{Exercise 6}
\textbf{Question:} For each of the following statements, fill in the blank with an easy-to-check criterion:\\
\textbf{(a)} M is a triangular number if and only if \rule{1cm}{0.15mm} is an odd square.\\
\textbf{(b)} N is an odd square if and only if \rule{1cm}{0.15mm} is a triangular number.\\
\textbf{(c)} Prove that your criteria in (a) and (b) are correct.\\
\\\textbf{Solution:}\\
\textbf{(a)} M is a triangular number if and only if $8M+1$ is an odd square.\\
\textbf{(b)} N is an odd square if and only if $\dfrac{N-1}{8}$ is a triangular number.\\
\textbf{(c)} Let us start with (a). Let us assume that the number $M$ is a triangular number. This fact is equivalent to the fact there exist a natural or zero number $n\in \mathbb{N}\cup \{0\}$ such that $M = \dfrac{n(n+1)}{2}$. Let us complete the square of the right side of the equation.
\begin{equation*}
M = \dfrac{n(n+1)}{2} = \dfrac{1}{2}(n+\dfrac{1}{2})^2-\dfrac{1}{8}
\end{equation*}
which is equivalent to the fact that
\begin{equation*}
8M+1=4(n+\dfrac{1}{2})^2=(2n+1)^2
\end{equation*}
Hence, the fact that $8M+1$ is an odd square. Now we need to give a proof in the opposite direction. Let us assume that the number $8M+1$ is an odd square. This fact is equivalent to the fact that $\exists n\in \mathbb{N}\cup\{0\}$ such that $8M+1 = (2n+1)^2$. Moreover, let us open the brackets of this equation to express $M$.
\begin{equation*}
M = \dfrac{1}{8}(2n+1)^2-\dfrac{1}{8}=\dfrac{n(n+1)}{2}
\end{equation*}
which says that $M$ is a triangular number.\\
\\ Now let us start (b). If $N$ is an odd square then, $\exists n\in \mathbb{N}\cup \{0\}$ such that $N = (2n+1)^2$. Then the number $N - 1 = 2n\cdot (2n+2)$ must be divisible by 8. Therefore, we have
\begin{equation*}
\dfrac{N-1}{8}=\dfrac{n(n+1)}{2}
\end{equation*}
which means that $\dfrac{N-1}{8}$ is a triangular number. If $\dfrac{N-1}{8}$ is a triangular number and there is a number $n\in \mathbb{N}\cup\{0\}$ such that $\dfrac{N-1}{8} = \dfrac{n(n+1)}{2}$. Hence, $N = (2n+1)^2$.
\end{document}
