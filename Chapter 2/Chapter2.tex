\documentclass{article}
\usepackage[utf8]{inputenc}
\usepackage{amsmath}
\usepackage{listings}
\usepackage{amssymb}
\usepackage{hyperref}
\usepackage{graphicx}

\title{%
  A Friendly Introduction to Number Theory \\
  \large Chapter 2: Pythagorean Triples \\
    Solutions}
\author{Hunter Matthews}
\date{09/03/2022}

\begin{document}

\maketitle
\newpage
\tableofcontents
\newpage
\section{Chapter 2: Pythagorean Triples}
\subsection{Exercise 1}
\textbf{Question:}\\
\textbf{(a)} We showed that in any primitive Pythagorean triple ($a, b, c$), either $a$ or $b$ is even. Use the same sort of argument to show that either $a$ or $b$ must be a multiple of 3.\\
\textbf{(b)} By examining the above list of primitive Pythagorean triples, make a guess about when $a, b$, or $c$ is a multiple of 5. Try to show that your guess is correct.\\
\\\textbf{Solution:}\\
\textbf{(a)} \emph{Lemma.} If the remainder of division of $x$ by $k$ is denoted $r$, then the remainder of division of $x^{2}$ by $k$ is denoted by $r^{2}$.\\\\ If $r$ denotes the remainder, $\exists m\in \mathbb{Z}\mid x = mk+r$. So we have $x^{2} = m^{2}k^{2}+2mkr+r^{2}$. This shows that $r^{2}$ and $x^{2}$ have the same remainders. Moreover, from the lemma, we know that the possible remainders of a square when divided by 3 are 0 and 1. This can be shown by a remainder table of division by 3 below.
\begin{displaymath}
\begin{array}{c|c}
x & x^2\\
\hline
0 & 0\\
1 & 1\\
2 & 1
\end{array}
\end{displaymath}
If we assume that $a$ and $b$ are not divisible by 3, then the remainder of $a^{2}$ when divided by 3 is equal to 1. The same works for $b^{2}$. However, we know this cannot be possible since $c^{2}$ should have a remainder of 2. Therefore, we get a contradiction and at least one of either $a$ or $b$ is divisible by 3.\\
\\\textbf{(b)} My guess would be that exactly one number from the PPT should be divisible by 5. To show this, let us build a table of division by 5.
\begin{displaymath}
\begin{array}{c|c}
x & x^2\\
\hline
0 & 0\\
1 & 1\\
2 & 4\\
3 & 4\\
4 & 1
\end{array}
\end{displaymath}
Therefore, the only possible triple of remainders would be
\begin{displaymath}
\begin{array}{c|c|c}
a^{2} & b^{2} & c^{2}\\
\hline
0 & 0 & 0\\
0 & 1 & 1\\
0 & 4 & 4\\
1 & 0 & 1\\
1 & 4 & 0\\
4 & 0 & 4\\
4 & 1 & 0
\end{array}
\end{displaymath}
With there being exactly one remainder zero in each of them, our assumption is proved.\\
\newpage
\subsection{Exercise 2}
\textbf{Question:} A nonzero integer $d$ is said to divide an integer $m$ if $m = dk$ for some number $k$. Show that if $d$ divides both $m$ and $n$, then $d$ also divides $m-n$ and $m+n$.\\
\\\textbf{Solution:}\\
If $m\mid d$ $\exists k_{1}\in \mathbb{Z}\mid m = k_{1}d$. If $n\mid d$ $\exists k_{2}\in \mathbb{Z}\mid n = k_{2}d$. Therefore, we have that $m + n = k_{1}d+k_{2}d = (k_{1}+k_{2})d$ and $m-n = k_{1}d-k_{2}d = (k_{1}-k_{2})d$. Since we know $k_{1}+k_{2}\in \mathbb{Z}$ and $k_{1}-k_{2}\in \mathbb{Z}$, then $m+n\mid d$ and $m-n\mid d$.
\newpage
\subsection{Exercise 3}
\textbf{Question:} For each of the following questions, begin by compiling some data; next, examine the data and formulate a conjecture; and finally, try to prove your conjecture is correct.\\
\\\textbf{(a)} Which odd numbers $a$ can appear in primitive Pythagorean triple ($a,b,c$)?\\
\\\textbf{(b)} Which even numbers $b$ can appear in a primitive Pythagorean triple ($a,b,c$)?\\
\\\textbf{(c)} Which numbers $c$ can appear in a primitive Pythagorean triple ($a,b,c$)?\\
\\\textbf{Solution:}\\
\textbf{(a)} Any odd number can exist as the $a$ in a primitive Pythagorean triple. In order to find such a triple, we can just let $t = a$ and $s = 1$ in the Pythagorean Triples Theorem. This gives us the following primitive Pythagorean Triple $$(a,\dfrac{(a^{2}-1)}{2},(\dfrac{(a^{2}+1)}{2})$$ \\
\\\textbf{(b)} Looking at the table on page 14, it seems as though $b$ is a multiple of $4$. We know that $b = \dfrac{(s^{2}-t^{2})}{2}$ with $s$ and $t$ both being odd. This means we can write $s = 2m+1$ and $t = 2n+1$. If we multiply things out, we get 
\begin{equation*}
\begin{split}
b = \dfrac{(2m+1)^{2}-(2n+1)^{2}}{2}
\\ = 2m^{2}+2m-2n^{2}-2n
\\ = 2m(m+1)-2n(n+1)
\end{split}
\end{equation*}
Notice how $m(m+1)$ and $n(n+1)$ must both be even. Hence, we have that $b\mid 4$.\\
\\ However, if $b\mid 4$, then we can write $b$ as $b = 2^{r}B$ for $B$ being odd and $r\geq 2$. We can try to find values of $s$ and $t$ such that $b = \dfrac{(s^{2}-t^{2})}{2}$. We can factor this as $$(s-t)(s+t)=2^{b}=2^{r+1}B$$
Now we know that both $s-t$ and $s+t$ must be even because $s$ and $t$ are odd. So we try
\begin{equation*}
\begin{split}
s-t=2^{r}
\\ s+t=2B
\end{split}
\end{equation*}
If we solve for $s$ and $t$, we get that $s = 2^{r-1}+B$ and $t = -2^{r-1}+B$. Since $B$ is odd and $r\geq 2$, we know that $s$ and $t$ must also be odd.
\begin{equation*}
\begin{split}
a = st = B^{2}-2^{2r-2}
\\ b = \dfrac{s^{2}-t^{2}}{2} = 2^{r}B
\\ c = \dfrac{s^{2}+t^{2}}{2} = B^{2}+2^{2r-2}
\end{split}
\end{equation*}
\\\textbf{(c)} This part was particularly difficult. I did not attempt to solve this. 
\newpage
\subsection{Exercise 4}
\textbf{Question:} In our list of examples, we have the two primitive Pythagorean triples $33^{2}+56^{2}=65^{2}$ and $16^{2}+63^{2}=65^{2}$. Find at least one more example of two primitive Pythagorean triples with the same value of $c$. Can you find three primitive Pythagorean triples with the same $c$? Can you find more than three?\\
\\\textbf{Solution:}\\
One example of two primitive Pythagorean triples with the same value for $c$ is $13^{2}+84^{2} = 85^{2}$ and $36^{2}+77^{2} = 85^{2}$. A primitve Pythagorean triple with 4 triples is
\begin{equation*}
\begin{split}
1105^2
&=47^2+1104^2\\
&=264^2+1073^2\\
&=576^2+943^2\\
&=744^2+817^2
\end{split}
\end{equation*}
\newpage
\subsection{Exercise 5}
\textbf{Question:} We have seen that the nth triangular number $T_n$ is given by the formula
\begin{equation*}
T_n = 1+2+3+\cdot\cdot\cdot+n=\dfrac{n(n+1)}{2}
\end{equation*}
The first few triangular numbers are 1, 3, 6, and 10. In the list of the first few Pythagorean triples ($a, b, c$), we find (3, 4, 5), (5, 12, 13), (7, 24, 25), and (9, 40, 41). Notice that in each case, the value of $b$ is four times a triangular number\\
\\\textbf{(a)} Find a primitive Pythagorean triple ($a,b,c$) with $b = 4T_{5}$. Do the same for $b = 4T_{6}$ and $b = 4T_{7}$.\\
\\\textbf{(b)} Do you think that for every triangular number $T_{n}$, there is a primitive Pythagorean triple ($a,b,c$) with $b = 4T_{n}$? If you believe that this is true, then prove it. Otherwise, find some triangular number for which it is not true.\\
\\\textbf{Solution:}\\ 
\textbf{(a)} We will build a table for the first 4 such triples.
\begin{displaymath}
\begin{array}{c|c|c|c|c|c}
n & a & b & c & s & t\\
\hline
1 & 3 & 4 & 5 & 3 & 1\\
2 & 5 & 12 & 13 & 5 & 1\\
3 & 7 & 24 & 25 & 7 & 1\\
4 & 9 & 40 & 41 & 9 & 1
\end{array}
\end{displaymath}
Our assumption would be that such triples can be found of the form $(2n+1)$, $(2n(n+1))$, and $2n^{2}+2n+1)$. We can check with the continuation of the table from above.
\begin{displaymath}
\begin{array}{c|c|c|c|c|c}
n & a & b & c & s & t\\
\hline
1 & 3 & 4 & 5 & 3 & 1\\
2 & 5 & 12 & 13 & 5 & 1\\
3 & 7 & 24 & 25 & 7 & 1\\
4 & 9 & 40 & 41 & 9 & 1\\
5 & 11 & 60 & 61 & 11 & 1\\
6 & 13 & 84 & 89 & 13 & 1\\
7 & 15 & 112 & 113 & 15 & 1
\end{array}
\end{displaymath}
\textbf{(b)} Let us show that the triple $((2n+1),(2n(n+1)), 2n^{2}+2n+1)$ are primitive Pythagorean triples. This triple is PPT since it is triple for $s = 2n+1$ and $t=1$. We can check this by the following
\begin{equation*}
\begin{split}
(2n+1)^2+4n^2(n+1)^2 = 4n^4+8n^3+8n^2+4n+1
\\(2n^2+2n+1)^2=4n^4+8n^3+8n^2+4n+1
\\(2n+1)^2+4n^2(n+1)^2=(2n^2+2n+1)^2
\end{split}
\end{equation*}
$2n+1$ is co-prime with $n,n+1$, and $2$. Therefore, $2n+1$ and $2n(n+1)$ are co-prime and this triple is a primitive Pythagorean triple.
\newpage
\subsection{Exercise 6}
\textbf{Question:} If you look at the table of primitive Pythagorean triples in this chapter, you will see many triples in which $c$ is 2 greater than $a$. For example, the triples ($3, 4, 5$), ($15, 8, 17$), ($35, 12, 37$), and ($63, 16, 65$) all have this property\\
\\\textbf{(a)} Find two more primitive Pythagorean triples ($a,b,c$) having $c = a+2$.\\
\\\textbf{(b)} Find a primitive Pythagorean triple ($a,b,c$) having $c = a+2$ and $c > 1000$.\\
\\\textbf{(c)} Try to find a formula that describes all primitive Pythagorean triples ($a,b,c$) having $c = a+2$.\\
\\\textbf{Solution:}\\
\textbf{(a)} We can find values $s$ and $t$ for known triples
\begin{displaymath}
\begin{array}{c|c|c|c|c|c}
n & a & b & c & s & t\\
\hline
1 & 3 & 4 & 5 & 3 & 1\\
2 & 15 & 8 & 17 & 5 & 3\\
3 & 35 & 12 & 37 & 7 & 5\\
4 & 63 & 16 & 65 & 9 & 7
\end{array}
\end{displaymath}
We can assume that such triples can be of the form ($4n^2-1,4n,4n^2+1$)
\begin{displaymath}
\begin{array}{c|c|c|c|c|c}
n & a & b & c & s & t\\
\hline
1 & 3 & 4 & 5 & 3 & 1\\
2 & 15 & 8 & 17 & 5 & 3\\
3 & 35 & 12 & 37 & 7 & 5\\
4 & 63 & 16 & 65 & 9 & 7\\
5 & 99 & 20 & 101 & 11 & 9\\
6 & 143 & 24 & 145 & 13 & 11
\end{array}
\end{displaymath}
\textbf{(b)}
\begin{displaymath}
\begin{array}{c|c|c|c|c|c}
n & a & b & c & s & t\\
\hline
1 & 3 & 4 & 5 & 3 & 1\\
2 & 15 & 8 & 17 & 5 & 3\\
3 & 35 & 12 & 37 & 7 & 5\\
4 & 63 & 16 & 65 & 9 & 7\\
5 & 99 & 20 & 101 & 11 & 9\\
6 & 143 & 24 & 145 & 13 & 11\\
16 & 1023 & 64 & 1025 & 33 & 31\\
50 & 9999 & 200 & 10001 & 101 & 99
\end{array}
\end{displaymath}
\textbf{(c)} If we look at the expressions for $a = st$ and $c = \dfrac{s^2+t^2}{2}$ and notice that $c-a=2=\dfrac{(s-t)^2}{2}$ we can see that $s = t+2$. Therefore, our formula gives the complete sequence of such numbers.
\newpage
\subsection{Exercise 7}
\textbf{Question:} For each primitive Pythagorean tripe $(a,b,c)$ in the table in this chapter, compute the quantity $2c-2a$. Do these values seem to have some special form? Try to prove that your observation is true for all primitive Pythagorean triples.\\
\\\textbf{Solution:}\\
First, let us compute $2c-2a$ based off the PPT table in Chapter 2.
\begin{displaymath}
\begin{array}{c|c|c|c|}
a & b & c & 2c-2a\\
\hline
3&4&5&4\\
5&12&13&16\\
7&24&25&36\\
9&40&41&64\\
15&8&17&4\\
21&20&29&16\\
35&12&37&4\\
45&28&53&16\\
63&16&65&4
\end{array}
\end{displaymath}
Looking for patterns in the $2c-2a$ column, it appears that they all are perfect squares. We can prove this using the Pythagorean Triples Theorem, which we know says $a = st$ and $c = \dfrac{(s^{2}+t^{2})}{2}$. So we have
\begin{equation*}
2c-2a = (s^{2}+t^{2})-2st = (s-t)^{2}
\end{equation*}
Thus proving that $2c-2a$ is always a perfect square.
\newpage
\subsection{Exercise 8}
\textbf{Question:} Let $m$ and $n$ be numbers that differ by 2, and write the sum $\dfrac{1}{m}+\dfrac{1}{n}$ as a fraction in lowest terms. For example, $\dfrac{1}{2}+\dfrac{1}{4} = \dfrac{3}{4}$ and $\dfrac{1}{3}+\dfrac{1}{5}=\dfrac{8}{15}$.\\
\\\textbf{(a)} Compute the next three examples.\\
\\\textbf{(b)} Examine the numerators and denominators of the fractions in $(a)$ and compare them with the table of Pythagorean triples. Formulate a conjectured about such fractions.\\
\\\textbf{(c)} Prove your conjecture is correct.\\
\\\textbf{Solution:}\\
\textbf{(a)}
\begin{displaymath}
\begin{array}{c|c|c|c|c}
n & m & 1/n + 1/m & num & den\\
\hline
2 & 4 & 3/4 & 3 & 4\\
3 & 5 & 8/15 & 8 & 15\\
4 & 6 & 5/12 & 5 & 12\\
5 & 7 & 12/35 & 12 & 35\\
6 & 8 & 7/24 & 7 & 24
\end{array}
\end{displaymath}
\textbf{(b)}
\begin{displaymath}
\begin{array}{c|c|c|c|c|c|c|c|c|c}
n & m & 1/n + 1/m & num & den & a & b & c & s & t\\
\hline
2 & 4 & 3/4 & 3 & 4 & 3 & 4 & 5 & 3 & 1\\
3 & 5 & 8/15 & 8 & 15 & 15 & 8 & 17 & 5 & 3\\
4 & 6 & 5/12 & 5 & 12 & 5 & 12 & 13 & 5 & 1\\
5 & 7 & 12/35 & 12 & 35 & 35 & 12 & 37 & 7 & 5\\
6 & 8 & 7/24 & 7 & 24 & 7 & 24 & 25 & 7 & 1
\end{array}
\end{displaymath}
Let our conjecture denoted the following. If $n$ is even then $n = 2k$, then $s = 2k+1$, $t = 1$ and
\begin{equation*}
\dfrac{1}{n}+\dfrac{1}{n+2}=\dfrac{s}{\dfrac{s^2-1}{2}}
\end{equation*}
If $n$ is odd then $n = 2k+1$, then $s = 2k+3$, $t = 2k+1$ and
\begin{equation*}
\dfrac{1}{n}+\dfrac{1}{n+2}=\dfrac{\dfrac{s^2-t^2}{2}}{st}
\end{equation*}
Hence, in both cases, we can form a Pythagorean triple from numerator and denominator.\\
\\\textbf{(c)} In the even case
\begin{equation*}
\dfrac{1}{2k}+\dfrac{1}{2k+2}=\dfrac{2(2k+1)}{4k(k+1)}=\dfrac{2(2k+1)}{(2k+1)^{2}-1}=\dfrac{s}{\dfrac{s^2-1}{2}}
\end{equation*}
In the odd case
\begin{equation*}
\dfrac{1}{2k+1}+\dfrac{1}{2k+3}=\dfrac{4(k+1)}{(2k+3)(2k+1)}=\dfrac{\dfrac{(2k+3)^{2}-(2k+1)^{2}}{2}}{(2k+3)(2k+1)}=\dfrac{\dfrac{s^2-t^2}{2}}{st}
\end{equation*}
\newpage
\subsection{Exercise 9}
\textbf{Question:} Answer the following questions related to the Babylonian number system.\\
\textbf{(a)} Read about the Babylonian number system and write a short description, including the symbols for the numbers 1 to 10 and the multiples of 10 from 20 to 50.\\
\\\textbf{(b)} Read about the Babylonian tablet called Plimpton 322 and write a brief report, including its approximate date of origin.\\
\\\textbf{(c)} The second and third columns of Plimpton 322 give pairs of integers ($a, c$) having the property that $c^{2}-a^{2}$ is a perfect square. Convert some of these pairs from Babylonian numbers to decimal numbers and compute the value of b so that ($a, b, c$) is a Pythagorean triple.\\
\\\textbf{Solution:}\\
\textbf{(a)} The Babylonians used a sexagesimal positional numerical system. Essentially, this means that they counted in base 60, using 59 different symbols (59 because we are not counting the zero) and that the position of a digit in a number is nothing but the multiplier of the relative power of 60. More information about the Babylonian number system can be found \href{https://mathshistory.st-andrews.ac.uk/HistTopics/Babylonian_numerals/}{here}. The symbols for the number 1-10, and the multiples of 10 from 20 to 50 are as follows\\
\includegraphics[width=\linewidth]{bab.png}
\\\textbf{(b,c)} You can read about the Babylonian tablet called Plimpton 322 and about the Plimpton 322 tablet containing pairs $(a,c)$ having the property $c^{2}-a^{2}$ letting you compute Pythagorean triples. You can read about it \href{https://personal.math.ubc.ca/~cass/courses/m446-03/pl322/pl322.html}{here}.\\
\end{document}
